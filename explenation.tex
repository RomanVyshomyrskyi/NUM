\documentclass[a4paper,12pt]{article}
\usepackage{amsmath,amsfonts}
\usepackage{graphicx}
\usepackage{float}
\usepackage{listings}
\usepackage{xcolor}
\usepackage{caption}
\usepackage{hyperref}
\usepackage{tocloft}

\lstset{
  language=R,
  basicstyle=\ttfamily\small,
  keywordstyle=\color{blue},
  commentstyle=\color{gray},
  stringstyle=\color{orange},
  numbers=left,
  numberstyle=\tiny,
  stepnumber=1,
  numbersep=5pt,
  frame=single,
  breaklines=true,
  tabsize=2,
  captionpos=b
}

\title{Test NUM}
\author{}
\date{}

\begin{document}

\maketitle
\tableofcontents
\newpage

\section{Exact Analytical Value}
The exact value of the integral is known and serves as a benchmark for evaluating the numerical results:
\[ I_\text{exact} = \ln(2) \]
In R, this is computed using:

\begin{lstlisting}[caption=Exact value of the integral]
I_analyticke <- log(2)
\end{lstlisting}

\section{Midpoint Rule Method}
The midpoint rule approximates the integral of a function over an interval $[a, b]$ by dividing it into $n$ equally sized subintervals of width $h = (b - a) / n$. For each subinterval, the function value at the midpoint is multiplied by the width:
\[ I_N = \sum_{i=0}^{n-1} h \cdot f\left(a + i h + \frac{h}{2}\right) \]
This method is particularly effective for smooth, continuous functions and is simple to implement. In R:

\begin{lstlisting}[caption=Midpoint rule function in R]
f <- function(x) {
  1 / x
}

midpoint_integral <- function(f, a, b, n) {
  h <- (b - a) / n
  I <- 0

  for (i in 0:(n - 1)) {
    xi <- a + i * h + h / 2
    Si <- h * f(xi)
    I <- I + Si
  }
  return(I)
}
\end{lstlisting}
\section{Error Computation}
To evaluate how the midpoint approximation converges, we compute the error for increasing values of $n = 2^i$, with $i$ ranging from 1 to 20. The error is computed as:
\[ y(i) = I_N(i) - I_\text{exact} \]
This shows how the numerical result deviates from the true value.

\begin{lstlisting}[caption=Compute midpoint error for i = 1 to 20]
i_val <- 1:20
y_val <- numeric(length(i_val))

for (j in seq_along(i_val)) {
  i <- i_val[j]
  n <- 2^i
  I_numericke <- midpoint_integral(f, 1, 2, n)
  y_val[j] <- I_numericke - I_analyticke
  if (j>1 && abs(y_val[j]) > abs(y_val[j-1])) {
    print(j)
  }
}
\end{lstlisting}

\section{Visualizing Error Behavior}
The error is plotted as a function of $i$. The error should generally decrease with increasing $n$:

\begin{lstlisting}[caption=Plot the numerical error]
plot(i_val, y_val, type = "b", pch = 19, col = "blue",
     xlab = "i", ylab = expression(y(i) == I[N](i) - I[A]))
abline(h = 0, col = "red", lty = 2)
\end{lstlisting}
\newpage
\section{Least Squares Approximation of the Error Function}
We want to approximate the numerical error $y(i)$ using a linear combination of basis functions:
\[ y(i) \approx c_0 \cdot \varphi_0(i) + c_1 \cdot \varphi_1(i) + c_2 \cdot \varphi_2(i) + c_3 \cdot \varphi_3(i) \]
where:
\begin{itemize}
  \item $\varphi_0(i) = 1$
  \item $\varphi_1(i) = \frac{1}{2^i}$ (exponential decay)
  \item $\varphi_2(i) = \frac{1}{i}$ (harmonic decay)
  \item $\varphi_3(i) = \frac{1}{i^2}$ (polynomial decay)
\end{itemize}

We construct a design matrix $A$ where $A[i,j] = \varphi_j(i)$, and solve the normal equations:
\[ (A^T A)\vec{c} = A^T \vec{y} \]

\begin{lstlisting}[caption=Least squares approximation function]
least_squares_approx <- function(x, y, basis_functions) {
  if(length(x) != length(y)) stop("x and y must have the same length")

  m <- length(x)
  n <- length(basis_functions)

  A <- matrix(0, nrow = m, ncol = n)
  for (j in 1:n) {
    A[, j] <- basis_functions[[j]](x)
  }
  coeff <- solve(t(A) %*% A, t(A) %*% y)
  return(as.vector(coeff))
}
\end{lstlisting}

\section{Computing and Interpreting the Coefficients}
The coefficients represent the best-fit parameters for the chosen basis functions to approximate the error behavior.

\begin{lstlisting}[caption=Define basis functions and compute coefficients]
poly_basis <- list(
  function(x) 1,
  function(x) 1 / (2^x),
  function(x) 1 / x,
  function(x) 1 / (x^2)
)

coeff_poly <- least_squares_approx(i_val, y_val, poly_basis)
print(coeff_poly)
\end{lstlisting}

\section{Evaluation and Plotting of the Approximation}
The approximated function is evaluated at each point $i$:

\begin{lstlisting}[caption=Evaluate and plot the approximation]
aprox_y_val <- sapply(i_val, function(xi) {
  sum(mapply(function(c, phi) c * phi(xi), coeff_poly, poly_basis))
})

lines(i_val, aprox_y_val, col = "darkgreen", lwd = 2)
\end{lstlisting}

\end{document}